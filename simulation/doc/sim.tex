\documentclass[a4paper,11pt]{article}
\usepackage{a4wide}


\usepackage{amsmath}

\begin{document}

% We want to simulate sets of reads mapping to a set, $L$, of loci for a
% population, $C$, of cells. The number of reads per locus will vary with
% locus and with cell.  To ensure biological relevance, we will base our
% simulation on reads and SNVs from an observed (subsetted) experimental
% data set comprising a bulk RNAseq data.
% Each locus $l\in L$ comprise a site $S$ and a site $G$.

\section{Notation}
\label{sec:notation}

\subsection{Loci and Cells}
\label{sec:loci-cells}

We will generate reads for a set $L$ of loci in a set $C$ of cells. A
locus in $L$ is defined as comprising two sites $S$ and $G$, close
enough to be covered by a (paired-end) read. The cells in $C$ are
assumed to form a related population.


\subsection{States of $G$}
\label{sec:states-g}


For a locus $l\in L$, the genome states of $G$ are referred to as
$g1/g2$, where $g1$ is the nucleotide state on allele $1$ and $g2$ is
the state on allele $2$.  $G$ will always be heterozygotic and,
moreover, the assignment of $g1$ and $g2$ to maternal and paternal
alleles are made such that $g1=R$ and $g2=A$, where $R$ is the
reference allele, and $A$ is the alternative
allele. This assignment is consistent within
$l$ and over $C$.

\subsection{States of $S$}
\label{sec:states-s}

The genome states of $S$ are similarly referred to as $s1/s2$, where
$s1$ and $s2$ are located on alleles 1 and 2, respectively, as
determined by the states of $G$ above. The states of $S$ are
determined by the simulation: If $S$ is a generated as a sSNV in the
population, the state of $S$ can be homozygotic or heterozygotic for
individual cells, while if $S$ is not an sSNV, all cells will be
homozygotic for $S$.  In the former case, all heterozygotic cells will
will have the same of two mutually exclusive possible genotypes, $(i)$
$s1=R$ and $s2=A$ or $(ii)$ $s1=A$ and $s2=R$, where $R$ is the
reference allele found in bulk and $A$ is a specified alternative
state. For simplicity of explanation, we will w.l.o.g. in the
following text assume case $(i)$ when discussing a sSNV locus. For all
homozygotic loci, we always have $s1=s2=R$. The described setup is
consistent within $l$, but may, as indicated, vary over $C$.

\subsection{Haplotypes and genotypes}
\label{sec:haplotypes-genotypes}

The haplotype for each allele, with respect to $S$ and $G$ will be
referred to as a tuple comprising the state of $S$ and the state of
$G$, in order, e.g., the haplotype $AR$ for allele
$i$ means that $s_i=A$ and $g_i=R$.

Under our assumptions above, there are only two possible $genotypes$,
i.e., pairs of haplotypes for a locus, either with a homozygote $S$:
$\{RR,RA\}$, or with a heterozygote $S$: $\{RR,AA\}$.

For a cell $c$ and a locus $l$, we will let $l(c)$ refer to the
genotype of $l$ in $c$ and let $l_i(c)$ refer to the haplotype of
allele $i$ och $l$ in $c$.


\subsection{Population structure}
\label{sec:population-structure}

To consistently model the distribution of mutated sSNVs among cells
for all loci, we need to describe a population structure of the cells.
A nice, general way to do this, allowing for any scenario we might
want to use, is to define the relationship between cells as a tree,
$T$, with inner nodes indexed, e.g., by numbers, and cells at the leaves. The
tree needs not be perfectly binary, but will typically have polytomies
-- in a very simple scenario, this could be a tree with two inner
nodes (clades) representing two clones of cells (plus a root
node). The inner node numbers can be viewed as indexing sets of
related cells (clades). Let $V$ be the set of numbers indexing inner vertices,
excluding the root of $T$.

When we, for a given sSNV locus, want to simulate which cells should
be heterozygotic for $S$, we simply draw a inner node number randomly
from $V$ and set all cells in the corresponding
set/clade to be heterozygotic; all remaining cells are set to be
homozygotic.  In our simple two-clone tree example, this would
correspond to making all cells in one clone heterozygotic and all
cells in the other clone homozygotic.

Since we use the same tree for all loci, all sSNVs will be consistent
with the population structure.

\emph{Notice that, by using a tree with a minimum clade size (e.g., no clade
  with less than $m$ cells will have a number), we can avoid generating cases
  where the mutated state occurs only in a single (or few) cell(s),
  while retaining some biological realism.}


\subsection{Empirical Reads sets}
\label{sec:reads-sets}


For each locus, $l$, we will work with four sets of reads covering a
locus with $S$ and $G$; The reads are derived from bulk-DNAseq of two
different cell populations, one of which one is heterozygous for $S$
and the other is homozygous for $S$; both are heterozygous for
$G$. The reads sets derived from each allele of the heterozygous
population are called $hetRef$ and $hetMut$, respectively. Similarly,
the reads sets from the homozygous population are called $homRef$ and
$homMut$. We will, w.l.o.g., assume that the allele assignments and
the $G$ and $S$ states are defined such that $hetRef=homMut=RR$.

\subsection{Empirical Coverage distribution}
\label{sec:cover-distr}
We will make use of a empirically determined frequency distribution,
$d_{COV}$, over reads coverage, i.e., the number of reads from each
allele in a locus. The distribution $d_{COV}$ is estimated over the
reads for all loci and cells from a single-cell DNAseq
experiment. This distribution is used i.i.d for all generated cells
and loci.


\section{Generation}
\label{sec:generation}
The locus-specific read sets will be used to generate cell population
scenarios comprising a locus $l$, which either is homozygotic for $S$
in all cells or is a sSNV, i.e., some cells are homozygotic for $S$
and other are heterozygotic, and a potential locus $l'$, which either
is homozygotic or heterozygotic for all cells; whether $S$ is a sSNV
is determined by a variable $SNV$.  The locus $l'$ will be used to
simulate possible alignments errors ($EAL$) when predicting the
genotype of $l$.
%
In addition, we will simulate dropout of reads ($DO$) in $l$ and
 $l'$, while empirical distributions are used for
allele-specific reads counts ($C$).

The strategy for generating reads is to use predefined probabilities
of $SNV$, $EAL$, $DO$ and the empirical distribution of $C$ for $l$
to step-by-step determine a count distribution $f_R$ of number of
reads to sample for cell $c\in C$, over the possible reads sets
$R=\{l_1(c), l_2(c), l'_1(c), l'_2(c)\} \subseteq \{hetRef, hetMut, homRef, homMut\}$
for $l$ in each cell $c$; notice that setting $f_R(h)=0$ for a reads
set $h$ is equivalent to excluding sampling from $h$. We then sample
reads following $f_R$ to simulate sequencing reads mapping to $l$.


\subsection{Generating $SNV$ and determining $R$ , or equivalently, assigning genotype to alleles of $l$ and $l'$}
\label{sec:generating-ssnv}


We first determine $SNV$ for a locus $l$, i.e., whether $l$ will have
an $sSNV$ in $S$ ($SNV=1$) or not ($SNV=0$). This can be done using
either of two strategies:
\begin{enumerate}
\item The locus $l$ will have a sSNV at $S$ ($SNV=1$) with a
  pre-defined probability $p_{SNV}$, which is i.i.d. over all
  loci. This strategy is biologically more realistic.
\item We use a pre-defined $f_{SNV}$, the frequency of sSNV loci in $L$, and then
  sample random subset of size $k=f_{SNV}|L|$ loci from $L$ and set
  $SNV=1$ for these loci only. This allows more precise control of the
  observed frequency of sSNV in the generated data.
\end{enumerate}

\begin{itemize}
\item If  $l$ has $SNV=1$
  , then
  \begin{enumerate}
  \item For all cells $c\in C$ initialize $l_i(c)$ and $l'_i(c) $ to homozygotic
    \begin{eqnarray*}
      l_1(c) &=& homRef \\
      l_2(c) &=& homMut \\
      l'_1(c) &=& homRef \\
      l'_2(c) &=& homMut 
    \end{eqnarray*}
  \item Sample a cell subset $V$ from the population tree
  \item For all cells $c\in V$, change $l(c)$ to heterozygotic
    \begin{eqnarray*}
      l_1(c) &=& hetRef \\
      l_2(c) &=& hetMut \\
    \end{eqnarray*}
  \end{enumerate}
\item Otherwise ($SNV=0$), for all cells $c\in C$, set $l(c)$ to homozygotic and $L'(c)$ to heterozygotic
  \begin{eqnarray*}
    l_1(c) &=& homRef \\
    l_2(c) &=& homMut \\
    l'_1(c) &=& hetRef \\
    l'_2(c) &=& hetMut 
  \end{eqnarray*}
\end{itemize}

\emph{Notice that we have chosen to let the value of $SNV$ also
  controls the read set assignment to $l'$ in such a way as to create
  problems for variant calling.}

\subsection{Generating EAL}
\label{sec:alignment-error}

Similar to the case of $SNV$ can determine whether a locus $l$ will
have an alignment error event ($EAL$) using two different strategies,
\begin{enumerate}
\item The locus $l$ will have an $EAL$ with a pre-defined probability
  $p_{EAL}$, which is i.i.d. over all loci.
\item We set $f_{EAL}$, the frequency of $EAL$ loci in $L$, and then
  sample a random subset of size  $k=f_{EAL}|L|$ loci from $L$ and set $EAL=1$
  for these loci, only.
\end{enumerate}

\begin{itemize}
\item If $EAL=1$ the initial values of $f_R$ for each $c\in C$ are set
  to:

\begin{eqnarray}
  f_R(l_1(c)) &=& 1\\
  f_R(l_2(c)) &=& 1\\
  f_R(l'_1(c)) &=& 1\\
  f_R(l'_2(c)) &=& 1.
\end{eqnarray}
\item Otherwise ($EAL=0$), the initial values of $f_R$ for each $c\in C$ are
set to (i.e., excluding sampling from $l'$):
\begin{eqnarray}
  f_R(l_1(c)) &=& 1\\
  f_R(l_2(c)) &=& 1\\
  f_R(l'_1(c)) &=& 0\\
  f_R(l'_2(c)) &=& 0.
\end{eqnarray}
\end{itemize}
\emph{Notice that presence of alignment errors is done on the \emph{population level}.}

\subsection{Read dropouts}
\label{sec:allelic-dropout}

Presence of DO in a locus is modeled on a per-cell basis.
%and both on the alleles of $l$ and $l'$. Hence, we define
$R_{DO} = \{l_1,l_2, l'_1, l'_2\}$. 

Also, the presence of DO in $c$ can be modeled in two ways:
\begin{enumerate}
\item For each cell $c \in C$ and for each allele $i\in R_{DO} = \{l_1,l_2, l'_1, l'_2\}$, set
  $f_R(i)=0$ with a pre-defined probability $p_{DO}$, which is
  i.i.d. over all cells and alleles.
\item Using a pre-defined frequency of DO in
  $C$$f_{DO}$, we sample a random subset, $C_{DO}$, of
  $k=f_{DO}|C|$ cells from $C$. For each $c\in
  C_{DO}$, we randomly sample an allele
  $l_{DO}$ from $R_{DO}=\{l_1,l_2\}$ and set
  $f_R(l_{DO})=0$.

  \emph{Notice that we in this case model $DO$ only on
    $l$; the reasoning for this is to simplify siulation while focusing
    on the problematic cases for variant calling.}
\end{enumerate}


\subsection{Generating COV, the number of reads per allele}
\label{sec:gener-numb-reads}


We sample the number of reads $COV(i)$ for each allele $i\in R$
i.i.d. from an empirical allele-specific coverage distribution
$d_{COV}$. Typically, we set a minimum value for COV (e.g., 10) to
avoid generating loci that will be filtered due to low coverage.

We then set

\begin{eqnarray*}
  f_R(l_1(c)) &=& COV(l_1) f_R(l_1(c))\\
  f_R(l_2(c)) &=& COV(l_2) f_R(l_2(c))\\
  f_R(l'_1(c)) &=& COV(l'_1) f_R(l'_1(c))\\
  f_R(l'_2(c)) &=& COV(l'_2) f_R(l'_2(c)).
\end{eqnarray*}

\emph{Notice that, if the 'input' $f_R(i)=0$ (to right of the equal sign), then it is guaranteed that also the 'output' $f_R(i)=0$ (to the left of the equal sign).}

\subsection{Sample reads from $f_R$}
\label{sec:sample-reads-from}
For cell $c$ and locus $l$, we will now create $r_{l,c}$, the set of
reads mapping to $l$ in $c$.

\begin{enumerate}
\item Initiate $r_{l,c}=\emptyset$.
\item for $i\in \{l_1(c),l_2(c),l'_1(c),l'_2(c)\}$
  \begin{enumerate}
  \item Repeat the following $f_R(i)$ times
    \begin{enumerate}
    \item Draw a read, $r$, from $i$ and add it to $r_{l,c}$, i.e., $r_{l,c}=r_{l,c}\cup \{r\}$.
    \end{enumerate}
  \end{enumerate}
\end{enumerate}

For each $c\in C$, the set of reads, $\{r_{*,c}$, is saved to a separate bam-file.

This concludes the description of the generative model and algorithm.

\end{document}

%%% Local Variables:
%%% mode: latex
%%% TeX-master: t
%%% End:
